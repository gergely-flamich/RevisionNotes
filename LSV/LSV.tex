\documentclass{article}

\usepackage{amsmath}
\usepackage{amssymb}
\usepackage{tikz}
\usetikzlibrary{arrows,automata}

\newcommand{\U}{\,\,\mathcal{U}\,\,}
\newcommand{\N}{\,\,\mathcal{N}\,\,}

\begin{document}

\section{Introduction}
\paragraph{Reliability:} continuity of service
\paragraph{Availability:} readiness for usage
\paragraph{Security:} e.g. preservation of confidentiality
\paragraph{Safety:} avoid catastrophic consequences

\paragraph{Fault prevention:} system does not contain faults
\paragraph{Fault tolerance:} system can recover from a system failure
\paragraph{Fail safety:} in case of failure, no fatal consequences can occur 

\paragraph{}
Preventing design bugs in hardware is important, because:
\begin{itemize}
\item costs are higher
\item bug fixes are usually not possible
\item quality expectations are higher
\item time a new product gets on the market severely impacts revenue
\end{itemize}

\paragraph{Validation:} Are we building the right thing?
\paragraph{Verification:} Are we building the thing right?

\paragraph{Problem with testing:} Can uncover faults, but doesn't guarantee
their absence.

\paragraph{Formal verification} can guarantee fault freeness.

\section{Model Checking}
\paragraph{Model:} formal abstraction of the system of interest
\paragraph{Constraints:} system requirements specified with precise logical statements 
A model checker automaticall checks the model against the constraints. If
something violates the constraints, an execution path is returned.

\paragraph{Examples of system properties}
\begin{itemize}
\item functional correctness: system behaving as expected?
\item reachability: can we end up in a deadlock?
\item safety: something ``bad'' never happens
\item liveness: something ``good'' will eventually happen
\item fairness: can, under certain conditions, an event occur repeatedly?
\item real-time: is the system acting in time?
\end{itemize}

Must ensure that model is \textbf{as simle as possible}: prevent state explosion.

\paragraph{Livelock:} a procass is always able to request resource, but always refused.
\paragraph{Unconditional fairness:} Every process executes infinitely often
\paragraph{Strong fairness:} Every process that can run, will run
\paragraph{Weak fairness:} Every process that can continuously run from a
certain point onwards, will execute infinitely often.

\section{PROMELA: Process Meta Language}
\paragraph{}
Spec language used in \textbf{Spin}.
Used to check \textbf{concurrent systems}:
\begin{itemize}
\item interleaved processes (stuff behaving independently)
\item shared global variables
\item synchronous/asynchronous channels
\end{itemize}

Can label certain parts of the model that are of interest and then assert stuff
only about those states.

Can check for desirable end states.

\paragraph{Checking liveness in Spin:} annotate program with
\texttt{progress[a-zA-Z0-9]*:} labels, and it will be checked if we pass through
the labels infinitely often.

\paragraph{Trace constraint:} can restrain the internal interleaving to only
perform certain kinds of transitions (i.e., some channels only send/receive one
kind of message, always alternating, etc.)
\\
A trace \textbf{synchronises} with the spec

\paragraph{Never claim:} define an undesired system behaviour and check if it
ever holds, and reports if it does.
\\
It \textit{succeeds} if it terminates of passes through its accept labels
infintely often. If no transition is possible in the never claim, the claim
automaton will stop search in that branch and backtrack.
A never claim is \textbf{interleaved} with the model spec. (It always executes
before the next transition)

\section{LTL: Linear Temporal Logic}
All valid LTL formulae may be derived from the grammar
\[
  \phi ::= \text{true} \,|\, \text{false} \,|\, p \,|\, \neg \,|\, \land \,|\,
  \lor \,|\, ( \phi ) \,|\, \diamond \phi \,|\, \square \phi \,|\, \phi \U \phi
\]
where $p$ is an \textbf{atomic proposition} (basically just a Boolean variable).
They are \textbf{simple} known facts about the model at a particular state.
\paragraph{Examples}
\begin{itemize}
\item Invariant: $\square p$
\item Reply: $p \Rightarrow \diamond q$
\item Guaranteed Reply: $\square(p \Rightarrow \N (\diamond q))$
\item Infinitely often: $\square\diamond p$
\item Eventually always: $\diamond\square p$
\item Exclusion: $\square(p \Rightarrow \neg q)$
\end{itemize}

\paragraph{Equivalences}
\begin{itemize}
\item $\neg\square p \equiv \diamond\neg p$
\item $\neg\diamond p \equiv \square \diamond\neg p$
\item $\square p \land q \equiv \square p \land \square q$ 
\item $\diamond p \lor q \equiv \diamond p \lor \diamond q$ 
\item $\diamond p \equiv \text{true} \U p$
\end{itemize}

\paragraph{Transition System} a TS is defined as $TS = (S, Act, T, I, AP, L)$
\begin{itemize}
\item $S$: set of all possible states (nodes of the multigraph)
\item $Act$: set of all possible transitions
\item $T \subseteq S \times Act \times S$: transition table (edges of the multigraph)
\item $I \subseteq S$: initial states
\item $AP$: set of atomic propositions
\item $L: S\to 2^{AP}$: labelling function (state labels)
\end{itemize}
A TS if \textbf{finite} iff $S, Act$ and $AP$ are finite.
A state satisfies a formula if its label set satisfies it, i.e.
\[
  s \vDash \phi \equiv L(s) \vDash \phi
\]
\paragraph{Definitions}
\begin{itemize}
\item \textbf{direct $\alpha$-successors of a state $s$}: $\text{Post}(s, \alpha) = \{s'
  \in S\,|\,(s,\alpha,s') \in T\}$ for $s\in S$ and $\alpha \in Act$. (States
  that are pointed to by an $\alpha$ arrow from $s$.)

\item \textbf{$\alpha$-predecessors of $s$}: $\text{Pre}(s, \alpha) = \{s'
  \in S\,|\,(s',\alpha,s) \in T\}$ for $s\in S$ and $\alpha \in Act$. (States
  that point to $s$ with an $\alpha$ arrow from $s$.)

\item \textbf{terminal state}: $s \in S $ is terminal iff $\text{Post}(s) = \emptyset$.

\item \textbf{finite execution fragment (FEF)}: for a TS, a FEF is
  \[
    \rho = s_0 \xrightarrow{\alpha_1} s_1 \xrightarrow{\alpha_2} \hdots
    \xrightarrow{\alpha_n} s_n
  \]
  where $s_i \in S$ and $\alpha_j \in Act$.
\item \textbf{maximal execution fragment (MEF)}: FEF ending in a terminal state or an
  infinite EF.

\item \textbf{initial execution fragment (IEF)}: EF whose first state is an
  initial state.

\item \textbf{execution of a TS}: IEF + MEF

\item \textbf{reachable state}: a state $s \in S$ is reachable iff
  \[
    \exists \rho. \quad \rho = s_0 \xrightarrow{\alpha_1} \hdots
    \xrightarrow{\alpha_n} s_n
  \]
  where $s_0 \in I$ and $s_n = s$ (you can transition into it in a finite amount
  of time).
\end{itemize}

\section{Data-Dependent Systems}
Want to deal with transitions dependent on some ''global state``.
In a TS: use nondeterminism (this doesn't scale well), or \textbf{conditional transitions}.

\paragraph{Conditional transition:} $g:\alpha$ where $\alpha$ is an action and
$g$ is a boolean condition (guard).

\paragraph{Definitions (bonkers)}
\begin{itemize}
\item \textbf{set of typed variables}: \texttt{Var}
\item \textbf{domain of a variable}: the type of $x \in \texttt{Var}$ is
  (inexplicably) called its domain, denoted $\text{dom}(x)$.

\item \textbf{evaluations}: The set of \textbf{evaluations} over \texttt{Var} is
  the set that contains how we allow variables to evaluate and is denoted
  $\text{Eval}(\texttt{Var})$. e.g. we might say
  that $x$ can evaluate to $1, 42$ and $y$ can evaluate to ``banana'',
  ``apple'', respectively. Then we
  can define $\eta_1(x)=1,\,\,\eta_1(y)=\text{``banana'''}\,\,\eta_2(x)=42,\,\,\eta_2(y)=$ ``apple'', and hence
  $\text{Eval}(\texttt{Var}) = \{\eta_1, \eta_2\}$.

\item \textbf{condition set}: the \textbf{condition set} over \texttt{Var} is a
  set of boolean conditions involving the elements of \texttt{Var}.

\item \textbf{Effect function}: indicates how variables evolve if we perform
  some action. Namely, $\text{Effect}: Act \times \text{Eval}(Var) \rightarrow
  \text{Eval}(\texttt{Var})$. e.g. if $\alpha$ represents incrementing $x$, and
  $\eta_i(x) = i$, then
  \[
    \text{Effect}(\alpha, \eta_i) = \eta_{i + 1}.
  \]
  Since this is a function, we can evaluate it:
  \[
    \text{Effect}(\alpha, \eta_i)(x) = \eta_{i + 1}(x) = i + 1.
  \]

\item \textbf{Program Graph}: a PG is $PG = (Loc, Act, \text{Effect}, C_\dagger, Loc_0,
  g_0)$, where
  \begin{itemize}
  \item $Loc$: set of locations, because the word state is not good anymore.
    (the justification being that the some location can be in different ``states'')
  \item $Act$: $Act$ is still good though, set of actions 
  \item $\text{Effect}$: see above
  \item $C_\dagger$: $C_\dagger \subseteq Loc \times
    \text{Cond}(\texttt{Var})\times Act \times Loc$, the conditional transition
    table. A generic element looks like
    \[
      l_1 \xrightarrow{g:\alpha} l_2
    \]
    i.e. if $g$ holds when we perform $\alpha$ in $l_1$, we transition into $l_2$.
  \item $Loc_0$: $Loc_0 \in Loc$ set of initial locations
  \item $g_0$: $g_0 \in \text{Cond}(\texttt{Var})$ is the initial condition
    (basically ensuring that our variables exist and have some initial value).
  \end{itemize}
\end{itemize}
A PG can be converted into a TS, by ``unfolding'' it, i.e. adding a state for
every location + evaluation combo, and a transition between them if we the guard
of the transition connecting the two locations is true.

\section{Parallelism}
\paragraph{}
TSs are good for \textbf{sequential} systems.
How do we model \textbf{parallel} systems?
Depends on how \textit{subsystem}s communticate: no communication, synchronous
comm., asynchronous comm.
\[
  TS = TS_1 \,\,||\,\, TS_2 \,\,|| \hdots ||\,\,TS_n
\]
where $||$ is the \textbf{parallel composition operator}, meaning that the
$TS_i$ execute at the same time.

\paragraph{Interleaving} is choosing the order of execution for the $TS_i$. Two
TSs interleaved is the Cartesian product of their states and union of their
action sets and atomic props, and is denoted $TS_1 ||| TS_2$, where $|||$ is
called the \textbf{interleaving operator}. Note: interleaving doesn't reflect
the expected behaviour of shared variables.\\
For programs with shared variables, we interleave PGs, and then unfold them.
\paragraph{Concurrency} is nondeterministic interleaving
\paragraph{Independent actions} are actions that can be interleaved, and their
effect doesn't change
\paragraph{Local variables} of $PG_1$ and $PG_2$ are $\texttt{Var}_1 \setminus
\texttt{Var}_2$ and vice versa.
\paragraph{Global Variables} are $\texttt{Var}_1 \cap
\texttt{Var}_2$.

\paragraph{Note:} Look at Lecture 8 - 10 for the graphs.
\end{document}