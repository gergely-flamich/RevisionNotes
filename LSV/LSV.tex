\documentclass{article}

\usepackage{amsmath}
\usepackage{amssymb}
\usepackage{tikz}
\usetikzlibrary{arrows,automata}

\newcommand{\U}{\,\,\mathcal{U}\,\,}
\newcommand{\N}{\,\,\mathcal{N}\,\,}

\begin{document}

\section{Introduction}
\paragraph{Reliability:} continuity of service
\paragraph{Availability:} readiness for usage
\paragraph{Security:} e.g. preservation of confidentiality
\paragraph{Safety:} avoid catastrophic consequences

\paragraph{Fault prevention:} system does not contain faults
\paragraph{Fault tolerance:} system can recover from a system failure
\paragraph{Fail safety:} in case of failure, no fatal consequences can occur 

\paragraph{}
Preventing design bugs in hardware is important, because:
\begin{itemize}
\item costs are higher
\item bug fixes are usually not possible
\item quality expectations are higher
\item time a new product gets on the market severely impacts revenue
\end{itemize}

\paragraph{Validation:} Are we building the right thing?
\paragraph{Verification:} Are we building the thing right?

\paragraph{Problem with testing:} Can uncover faults, but doesn't guarantee
their absence.

\paragraph{Formal verification} can guarantee fault freeness.

\section{Model Checking}
\paragraph{Model:} formal abstraction of the system of interest
\paragraph{Constraints:} system requirements specified with precise logical statements 
A model checker automaticall checks the model against the constraints. If
something violates the constraints, an execution path is returned.

\paragraph{Examples of system properties}
\begin{itemize}
\item functional correctness: system behaving as expected?
\item reachability: can we end up in a deadlock?
\item safety: something ``bad'' never happens
\item liveness: something ``good'' will eventually happen
\item fairness: can, under certain conditions, an event occur repeatedly?
\item real-time: is the system acting in time?
\end{itemize}

Must ensure that model is \textbf{as simle as possible}: prevent state explosion.

\paragraph{Livelock:} a procass is always able to request resource, but always refused.
\paragraph{Unconditional fairness:} Every process executes infinitely often.
($\square\diamond p$)
\paragraph{Strong fairness:} Every process that can run, will run.
$\square\diamond q \Rightarrow \square\diamond p$
\paragraph{Weak fairness:} Every process that can continuously run from a
certain point onwards, will execute infinitely often. $\diamond\square q \Rightarrow \square\diamond p$

\section{PROMELA: Process Meta Language}
\paragraph{}
Spec language used in \textbf{Spin}.
Used to check \textbf{concurrent systems}:
\begin{itemize}
\item interleaved processes (stuff behaving independently)
\item shared global variables
\item synchronous/asynchronous channels
\end{itemize}

Can label certain parts of the model that are of interest and then assert stuff
only about those states.

Can check for desirable end states.

\paragraph{Checking liveness in Spin:} annotate program with
\texttt{progress[a-zA-Z0-9]*:} labels, and it will be checked if we pass through
the labels infinitely often.

\paragraph{Trace constraint:} can restrain the internal interleaving to only
perform certain kinds of transitions (i.e., some channels only send/receive one
kind of message, always alternating, etc.)
\\
A trace \textbf{synchronises} with the spec

\paragraph{Never claim:} define an undesired system behaviour and check if it
ever holds, and reports if it does.
\\
It \textit{succeeds} if it terminates of passes through its accept labels
infintely often. If no transition is possible in the never claim, the claim
automaton will stop search in that branch and backtrack.
A never claim is \textbf{interleaved} with the model spec. (It always executes
before the next transition)

\section{LTL: Linear Temporal Logic}
All valid LTL formulae may be derived from the grammar
\[
  \phi ::= \text{true} \,|\, \text{false} \,|\, p \,|\, \neg \,|\, \land \,|\,
  \lor \,|\, ( \phi ) \,|\, \diamond \phi \,|\, \square \phi \,|\, \phi \U \phi
\]
where $p$ is an \textbf{atomic proposition} (basically just a Boolean variable).
They are \textbf{simple} known facts about the model at a particular state.
\paragraph{Note:} next can be denoted $X, \N$ or $\circ$.
\paragraph{Examples}
\begin{itemize}
\item Invariant: $\square p$
\item Reply: $p \Rightarrow \diamond q$
\item Guaranteed Reply: $\square(p \Rightarrow \N (\diamond q))$
\item Infinitely often: $\square\diamond p$
\item Eventually always: $\diamond\square p$
\item Exclusion: $\square(p \Rightarrow \neg q)$
\end{itemize}

\paragraph{Equivalences}
\begin{itemize}
\item $\neg\square p \equiv \diamond\neg p$
\item $\neg\diamond p \equiv \square \diamond\neg p$
\item $\square p \land q \equiv \square p \land \square q$ 
\item $\diamond p \lor q \equiv \diamond p \lor \diamond q$ 
\item $\diamond p \equiv \text{true} \U p$
\end{itemize}

\paragraph{Transition System} a TS is defined as $TS = (S, Act, T, I, AP, L)$
\begin{itemize}
\item $S$: set of all possible states (nodes of the multigraph)
\item $Act$: set of all possible transitions
\item $T \subseteq S \times Act \times S$: transition table (edges of the multigraph)
\item $I \subseteq S$: initial states
\item $AP$: set of atomic propositions
\item $L: S\to 2^{AP}$: labelling function (state labels)
\end{itemize}
A TS if \textbf{finite} iff $S, Act$ and $AP$ are finite.
A state satisfies a formula if its label set satisfies it, i.e.
\[
  s \vDash \phi \equiv L(s) \vDash \phi
\]
\paragraph{Definitions}
\begin{itemize}
\item \textbf{direct $\alpha$-successors of a state $s$}: $\text{Post}(s, \alpha) = \{s'
  \in S\,|\,(s,\alpha,s') \in T\}$ for $s\in S$ and $\alpha \in Act$. (States
  that are pointed to by an $\alpha$ arrow from $s$.)

\item \textbf{$\alpha$-predecessors of $s$}: $\text{Pre}(s, \alpha) = \{s'
  \in S\,|\,(s',\alpha,s) \in T\}$ for $s\in S$ and $\alpha \in Act$. (States
  that point to $s$ with an $\alpha$ arrow from $s$.)

\item \textbf{terminal state}: $s \in S $ is terminal iff $\text{Post}(s) = \emptyset$.

\item \textbf{finite execution fragment (FEF)}: for a TS, a FEF is
  \[
    \rho = s_0 \xrightarrow{\alpha_1} s_1 \xrightarrow{\alpha_2} \hdots
    \xrightarrow{\alpha_n} s_n
  \]
  where $s_i \in S$ and $\alpha_j \in Act$.
\item \textbf{maximal execution fragment (MEF)}: FEF ending in a terminal state or an
  infinite EF.

\item \textbf{initial execution fragment (IEF)}: EF whose first state is an
  initial state.

\item \textbf{execution of a TS}: IEF + MEF

\item \textbf{reachable state}: a state $s \in S$ is reachable iff
  \[
    \exists \rho. \quad \rho = s_0 \xrightarrow{\alpha_1} \hdots
    \xrightarrow{\alpha_n} s_n
  \]
  where $s_0 \in I$ and $s_n = s$ (you can transition into it in a finite amount
  of time).
\item \textbf{state satisfying LTL formula}: a state $s \in S$ satisfies an LTL
  claim $\phi$ iff all paths starting in $s$ satisfy it.
\item \textbf{TS satisfying LTL formula}: a TS satisfies and LTL claim iff all
  of its initial paths satisfy it. \textbf{Note:} since one path might satisfy
  a claim, and another might not, it is possible that a TS satisfies neither a
  claim nor its negation!
\end{itemize}

\section{Data-Dependent Systems}
Want to deal with transitions dependent on some ''global state``.
In a TS: use nondeterminism (this doesn't scale well), or \textbf{conditional transitions}.

\paragraph{Conditional transition:} $g:\alpha$ where $\alpha$ is an action and
$g$ is a boolean condition (guard).

\paragraph{Definitions (bonkers)}
\begin{itemize}
\item \textbf{set of typed variables}: \texttt{Var}
\item \textbf{domain of a variable}: the type of $x \in \texttt{Var}$ is
  (inexplicably) called its domain, denoted $\text{dom}(x)$.

\item \textbf{evaluations}: The set of \textbf{evaluations} over \texttt{Var} is
  the set that contains how we allow variables to evaluate and is denoted
  $\text{Eval}(\texttt{Var})$. e.g. we might say
  that $x$ can evaluate to $1, 42$ and $y$ can evaluate to ``banana'',
  ``apple'', respectively. Then we
  can define $\eta_1(x)=1,\,\,\eta_1(y)=\text{``banana'''}\,\,\eta_2(x)=42,\,\,\eta_2(y)=$ ``apple'', and hence
  $\text{Eval}(\texttt{Var}) = \{\eta_1, \eta_2\}$.

\item \textbf{condition set}: the \textbf{condition set} over \texttt{Var} is a
  set of boolean conditions involving the elements of \texttt{Var}.

\item \textbf{Effect function}: indicates how variables evolve if we perform
  some action. Namely, $\text{Effect}: Act \times \text{Eval}(Var) \rightarrow
  \text{Eval}(\texttt{Var})$. e.g. if $\alpha$ represents incrementing $x$, and
  $\eta_i(x) = i$, then
  \[
    \text{Effect}(\alpha, \eta_i) = \eta_{i + 1}.
  \]
  Since this is a function, we can evaluate it:
  \[
    \text{Effect}(\alpha, \eta_i)(x) = \eta_{i + 1}(x) = i + 1.
  \]

\item \textbf{Program Graph}: a PG is $PG = (Loc, Act, \text{Effect}, C_\dagger, Loc_0,
  g_0)$, where
  \begin{itemize}
  \item $Loc$: set of locations, because the word state is not good anymore.
    (the justification being that the some location can be in different ``states'')
  \item $Act$: $Act$ is still good though, set of actions 
  \item $\text{Effect}$: see above
  \item $C_\dagger$: $C_\dagger \subseteq Loc \times
    \text{Cond}(\texttt{Var})\times Act \times Loc$, the conditional transition
    table. A generic element looks like
    \[
      l_1 \xrightarrow{g:\alpha} l_2
    \]
    i.e. if $g$ holds when we perform $\alpha$ in $l_1$, we transition into $l_2$.
  \item $Loc_0$: $Loc_0 \in Loc$ set of initial locations
  \item $g_0$: $g_0 \in \text{Cond}(\texttt{Var})$ is the initial condition
    (basically ensuring that our variables exist and have some initial value).
  \end{itemize}
\end{itemize}
A PG can be converted into a TS, by ``unfolding'' it, i.e. adding a state for
every location + evaluation combo, and a transition between them if we the guard
of the transition connecting the two locations is true. Formally:
\[
  \frac{l_1 \xrightarrow{g:\alpha} l_2 \land \eta \vDash g}{\langle l_1, \eta
    \rangle \xrightarrow{\alpha} \langle l_2, \text{Effect}(\alpha, \eta) \rangle}
\]
\section{Parallelism}
\paragraph{}
TSs are good for \textbf{sequential} systems.
How do we model \textbf{parallel} systems?
Depends on how \textit{subsystem}s communticate: no communication, synchronous
comm., asynchronous comm.
\[
  TS = TS_1 \,\,||\,\, TS_2 \,\,|| \hdots ||\,\,TS_n
\]
where $||$ is the \textbf{parallel composition operator}, meaning that the
$TS_i$ execute at the same time.

\paragraph{Interleaving} is choosing the order of execution for the $TS_i$. Two
TSs interleaved is the Cartesian product of their states and union of their
action sets and atomic props, and is denoted $TS_1 ||| TS_2$, where $|||$ is
called the \textbf{interleaving operator}. Note: interleaving doesn't reflect
the expected behaviour of shared variables.\\
For programs with shared variables, we interleave PGs, and then unfold them.
\paragraph{Concurrency} is nondeterministic interleaving
\paragraph{Independent actions} are actions that can be interleaved, and their
effect doesn't change
\paragraph{Local variables} of $PG_1$ and $PG_2$ are $\texttt{Var}_1 \setminus
\texttt{Var}_2$ and vice versa.
\paragraph{Global Variables} are $\texttt{Var}_1 \cap
\texttt{Var}_2$.

\paragraph{Note:} Look at Lecture 8 - 10 for the graphs.

\subsection{Handshaking}
Processes interact at the same time through sychronous communication.
handshaking done through shared actions: for $TS_1$, $TS_2$ and $H \subseteq
Act_1 \cap Act_2$. Then
\[
  TS_1 ||_H TS_2 \equiv (S_1 \times S_2, Act_1 \cup Act_2, T, I_1 \times I_2,
  AP_1 \cup AP_2, L)
\]
where if an action is not in $H$, then we add two arrows for the two separate
state tuples, whereas if the action is in $H$, then the states connected by
$\alpha$ are merged. Formally:
\[
  \forall \alpha \not\in H. \quad \frac{s_1 \xrightarrow{\alpha} s_1'}{\langle
    s_1, s_2 \rangle \xrightarrow{\alpha} \langle
    s_1', s_2 \rangle} \quad\text{and}\quad \frac{s_2 \xrightarrow{\alpha} s_2'}{\langle
    s_1, s_2 \rangle \xrightarrow{\alpha} \langle
    s_1, s_2' \rangle} \quad\text{(interleaving)}
\]
and
\[
  \forall \alpha \in H. \quad \frac{s_1 \xrightarrow{\alpha} s_1' \land s_2 \xrightarrow{\alpha} s_2'}{\langle
    s_1, s_2 \rangle \xrightarrow{\alpha} \langle
    s_1', s_2' \rangle}  \quad\text{(synchronisation)}
\]

\paragraph{Note:} $TS_1 ||_\emptyset = TS_1 ||| TS_2$.

\section{Asynchronous communication}
Processes communicate via channels.
Many critical systems depend on time as well: train crossing, radiation machine,
space shuttles etc.
How to add time?
\paragraph{Bad Approach:} use global ``clock variable'' and synchronise all
transitions on it. Why it's bad: it's confusing, overly complex and error-prone.
\paragraph{Good Approach:} Timed Automata
\subsection{Timed Automata}
Extend a program graph with a finite set of \textbf{clock variables}, which can
only be \textbf{inspected} or \textbf{reset}. We can then use impose
\textbf{clock constraints}: guards and time a PG spends in a state.
\paragraph{definitons}
\begin{itemize}
\item \textbf{Timed Automaton}: $TA = PG + (C, Inv) $, where $C$ is the set of
  clocks and $Inv: Loc \rightarrow CC(C)$ where $CC(C)$ is the set of clock
  constraints.
\item \textbf{Clock Reset}: $l_1 \xrightarrow{g:\alpha, D} l_2$ everything as in
  the $PG$, but also reset all clocks to 0 in $D \subseteq C$.
\item \textbf{Timelock}: if we can't leave a state before one of its invariants
  expire (i.e. one of the clocks violates its time constraint).
\item \textbf{Interleaving and handshaking}: defined analogously to PGs, but
  when we synchronise, we union the clocks and the constraints too. 
\end{itemize}

\paragraph{Note:} See Lecture 11-13 for graphs.

\section{CTL: Computational Tree Logic}
want to express constraints about the paths: LTL talks about all executions, no
good.
Solution: CTL.
\paragraph{Definitions}
\begin{itemize}
\item \textbf{CTL formula} a valid CTL claim can be decomposed into
  \textbf{State formulae} and \textbf{Path formulae}. It can be derived from the
  following and only the following grammar:
  \begin{gather*}
    \Phi ::= \text{true} \,\,|\,\, a\,\,|\,\, \Phi \land \Phi \,\,|\,\, \neg
    \Phi\,\,|\,\, \exists \phi\,\,|\,\,\forall \phi\quad \textbf{(State Formulae)}\\
    \phi ::= \N \Phi\,\,|\,\, \Phi \U \Phi\quad \textbf{(Path Formulae)}
  \end{gather*}
  Other operators may be derived from the above.
  \\
  $\exists \phi$ holds in a state $s$ iff there is at least one path from $s$
  that satisfies $\phi$.
  \\
  $\forall \phi$ holds in a state iff all paths from it satisfy $\phi$
  \textbf{Note:} if the second argument of $\U$ never holds, it is false.
\item \textbf{Satisfaction Set}: $\text{Sat}_{TS}(\Phi) = \{s \in S\,|\,s\vDash
  \Phi\}$, i.e. the states in the TS that satisfy the state formula. (NOT just
  the initial states!)
\item \textbf{TS satisfies CTL claim}: if all the inital states satisfy the
  claim. Formally: $I \subseteq \text{Sat}_{TS}(\Phi)$.
\end{itemize}
\paragraph{Model checking a TS against a CTL claim:} recursively build all the
satisfaction sets starting from the terminal states (including the end-point of
loops), then check if $I \subseteq \text{Sat}_{TS}(\Phi)$.
\paragraph{Examples}
\begin{itemize}
\item $\exists\diamond P$: $P$ potentially holds (there is at least one path
  where it holds at least for one state). ``There will be a day when it will
  rain for some time''
\item $\exists\square P$: $P$ potentially always hold (there is at least one
  path where it always holds). ``There is a timeline where from now on it rains
  until the end of time''
\item $\forall\diamond P$: $P$ is inevitable (it will hold at some point for all
  paths). ``I will die at some point, no matter what''.
\item $\forall\square P$: $P$ is invariantly true (it holds from the point
  onwards for all paths). ``2 + 2 = 4''.
\item $\neg \exists \diamond \Phi = \forall \square \neg \Phi$
\item $\forall\diamond \Phi = \forall (\text{true} \U \Phi)$
\end{itemize}

\section{TCTL: Timed CTL}
Just like CTL, but with clocks added from timed automata.
Doesn't have $\N$, only have
\[
  \phi ::= \Phi \U^J \Phi
\]
where $J$ is a non-negative interval of time until which the first argument must hold.
\paragraph{Examples}
\begin{itemize}
\item $\exists\square^J \Phi$: there is a path where $\Phi$ holds during
  interval $J$. ``There is a timeline where it will continuously rain between 11:00 and 15:00
  tomorrow.''
\item $\forall\square^J \Phi$: $\Phi$ always holds in the interval. ``In any
  case I will sleep between midnight and 8 am''
\end{itemize}
\section{UPPAAL not examined}
\section{Petri Nets}
Graphical descriptions for distributed systems for behavioural analysis.
\paragraph{Definitions}
\begin{itemize}
\item \textbf{Petri Net}: $N = (P, T, G)$, where
  \begin{itemize}
  \item $P$ set of places (nodes)
  \item $T$ set of transitions 
  \item $G$ directed graph linking places and transitions. Formally: $G
    \subseteq (P \times T) \cup (T \times P)$
  \end{itemize}
\item \textbf{pre-transition regions}: $\text{pre}(t) = \{ p \in P \,|\, (p, t)
  \in G\}$, the set of places from which we can transition using $t$.
\item \textbf{post-transition regions}: $\text{post}(t) = \{ p \in P \,|\, (t, p)
  \in G\}$, the set of places to which we can transition using $t$.
\item \textbf{marking}: $M: P \rightarrow \mathbb{N}_0$, the number of tokens at
  a place.
\item \textbf{transition firing}: $t$ can fire under $M$ iff there is one token
  in every element in the pre-transition set of $t$. Formally, $\forall p \in
  \text{pre}(t)$, we have $M(p) \geq 1$. If a transition can fire, then we say
  it is \textbf{enabled}. A firing takes a token from each of its pre-set
  elements and moves one token to each element of its post-set.
\item \textbf{concurrently enabled transitions}: The set of all transitions that
  are enabled.
\item \textbf{reachable markings} let $N$ a Petri Net and $M$ a marking for it,
  then $\text{Reachable}(N, M) = \{M' | M' \,\,\text{is a marking for}\,\,N\}$,
  such that there is a sequence of transitions through which we get to $M'$.
  \\
  \textbf{Note}: a marking for which there is no enabled transition signals a \textbf{deadlock}.
\item \textbf{Reachability Graph}: the Petri Net equivalent of an unfolded
  Program Graph, i.e.: it models how a Petri net can evolve, where the nodes are
  the entire state current state (how many tokens are in every place) with the
  set of enabled transitions being the edges. \\
  \textbf{Note:} this may be an infinite graph.
\item \textbf{Overlapping Graph}: generalised Reachability Graph, where if we
  can have two markings where one marking $M$ has bigger values than some other
  marking $M'$, then we merge the two markings together.
\end{itemize}

\paragraph{Note}: reachability of a marking from another is
\textbf{undecidable}.
\paragraph{Note}: the reachability of an overlapping marking given a marking $M$
is decidable.
\end{document}